\documentclass{bioinfo}
\usepackage{algorithmic}
\usepackage{algorithm}

\begin{document}
\title[Supplemental file 1]{Supplemental file to the MPCC paper}

\section{Pearson$'$s Correlation Coefficient}
When we say Pearson$'$s correlation coefficient, we mean:

\begin{equation}
%r=\frac{n\sum x_iy_i-\sum x_i\sum y_i}{((n\sum x_i^2-(\sum x_i)^2)(n\sum y_i^2-(\sum y_i)^2)^(1/2)}
r=\frac{n\sum x_iy_i-\sum x_i\sum y_i}{((n\sum x_i^2-(\sum x_i)^2)(n\sum x_i^2-(\sum y_i)^2))}
\end{equation}
where $x$ and $y$ are vectors of length $n$, and $r$ is a real number between -1 and 1 inclusive.

\section{The Naive Version}

Given $A$, an arbitrary number of vectors of length $n$, and $B$, an arbitrary number of vectors of length $n$, 
we define the naive version like so:

\begin{algorithmic}[1]
\FOR{Each vector  $i$ in set of vectors $A$}
  \FOR{Each vector  $j$ in set of vectors $B$}
    \STATE $m=n$
    \STATE $SA_{ij}=SAA_{ij}=SB_{ij}=SBB_{ij}=SAB_{ij}=0$
    \FOR{Each element $k$  in vectors $i$ and $j$}
      \IF{element $k$ in vectors $i$ or $j$ missing}
        \STATE Decrement vector length: $m=m-1$
      \ELSE
        \STATE $SA_{ij}+=A_{ik}$
        \STATE $SB_{ij}+=B_{jk}$
        \STATE $SAA_{ij}+=A_{ik}A_{ik}$
        \STATE $SBB_{ij}+=B_{jk}B_{jk}$
        \STATE $SAB_{ij}+=A_{ik}B_{jk}$
      \ENDIF
    \ENDFOR 
    \STATE $R_{ij}=\frac{mSAB_{ij}-SA_{ij}SB_{ij}}{((mSAA_{ij}-(SA_{ij}SA_{ij})(mSBB_{ij}-(SB_{ij}SB_{ij}))}$
  \ENDFOR
\ENDFOR
\end{algorithmic}

\noindent where 
$SA_{ij}$,
$SAA_{ij}$, 
$SB_{ij}$, 
$SBB_{ij}$, and $SAB_{ij}$ 
are elements in the sums of the elements 
of vector $i$ in $A$, 
squared of vector $i$ in $A$, 
of vector $j$ in $B$, 
squared of vector $j$ in $B$, and
of vector $i$ in $A$ times the  elements of vector $j$ in $B$, respectively.
Note that $R_{ij}$ is the PCC of vector $i$ of $A$ and vector $j$ of $B$.

\section{The Matrix Version}

\begin{algorithmic}[1]

  \STATE $\sum A=\text{SGEMM(}A_Z,U_B\text{)}$
  \STATE $\sum A^2=\text{SGEMM(}A_Z^2,U_B\text{)}$
  \STATE $\sum B=\text{SGEMM(}B_Z,U_A\text{)}$
  \STATE $\sum B^2=\text{SGEMM(}B_Z^2,U_A\text{)}$
  \STATE $\sum AB=\text{SGEMM(}A_Z,B_Z\text{)}$
  
  \vspace{2mm}

  \STATE Compute $M\sum A^2$   (Element-wise multiplication)
  \STATE Compute $M\sum B^2$   (Element-wise multiplication)
  \STATE Compute $M\sum AB$   (Element-wise multiplication)
  
  \vspace{2mm}

  \STATE $N_0=M\sum AB - \sum A\sum B$   (Element-wise subtraction)
  \STATE $D_0=M\sum A^2 - \sum A\sum A$   (Element-wise subtraction)
  \STATE $D_1=M\sum B^2 - \sum B\sum B$   (Element-wise subtraction)

  \vspace{2mm}

  \STATE $D_2=D_0 D_1 $   (Element-wise multiplication)
  \STATE $D_3=\text{SQRT(}D_2\text{)}$   (Element-wise square root)

  \vspace{2mm}

  \STATE $P=N_0 / D_3$   (Element-wise division)
  
\end{algorithmic}

\noindent where  $A_Z$ is the set of vectors $A$ with a zero at every 
missing data location, $A_Z^2$ is the result of squaring every element 
of $A_Z$, $B_Z$ is the set of vectors $B$ with a zero at every 
missing data location, $B_Z^2$ is the result of squaring every element 
of $B_Z$, $M$ is a matrix whose $i, j$-th element is the number of 
elements in the $i$-th vector of $A$, and the $j$-th vector of $B$ 
such that there is no missing data in either $A$ or $B$, and $P$ is 
a matrix of PCC calculations. All the other variables are self explanatory.

\section{Missing Data Bit-masking} \label{MDBM}

To generate $M$ above, a bitmask for each vector in $A$ and for each vector in $B$, 
is first initialized to all zeros then set to one in every position where there is missing data.
Then, to find the number of positions where there is no missing data for a given vector $i$ 
in $A$ and $j$ in $B$, a bitwise OR is done between the the corresponding bitmasks, and the 
the number of zeros in the result is the size of the PCC calculation for those two vectors.

\end{document}